\documentclass{article}
\usepackage[polish]{babel}
\usepackage[utf8]{inputenc}
\usepackage{polski}
\usepackage[T1]{fontenc}
\frenchspacing
\usepackage{indentfirst}

\usepackage{graphicx}
\usepackage{verbatim} % used to display code
\usepackage{hyperref}
\usepackage{fullpage}
\usepackage[usenames,dvipsnames]{color}
\usepackage{float}
\usepackage{subfig}
\usepackage{tikz}
\usepackage{fancyvrb}
\usepackage{acronym}
\usepackage{amsthm} % Uuhhh yet another package
\VerbatimFootnotes % Required, otherwise verbatim does not work in footnotes!

\usepackage{listings}

\definecolor{Brown}{cmyk}{0,0.81,1,0.60}
\definecolor{OliveGreen}{cmyk}{0.64,0,0.95,0.40}
\definecolor{CadetBlue}{cmyk}{0.62,0.57,0.23,0}

\lstset{
language=C++,
basicstyle=\ttfamily,
keywordstyle=\color{OliveGreen},
commentstyle=\color{gray},
numbers=left,
numberstyle=\tiny,
stepnumber=1,
numbersep=5pt,
frame=none,
tabsize=8,
captionpos=b,
breaklines=true,
breakatwhitespace=false,
showspaces=false,
showtabs=false,
columns=flexible,
inputencoding=utf8x,
extendedchars=\true,
literate={ą}{{\k{a}}}1
{Ą}{{\k{A}}}1
{ę}{{\k{e}}}1
{Ę}{{\k{E}}}1
{ó}{{\'o}}1
{Ó}{{\'O}}1
{ś}{{\'s}}1
{Ś}{{\'S}}1
{ł}{{\l{}}}1
{Ł}{{\L{}}}1
{ż}{{\.z}}1
{Ż}{{\.Z}}1
{ź}{{\'z}}1
{Ź}{{\'Z}}1
{ć}{{\'c}}1
{Ć}{{\'C}}1
{ń}{{\'n}}1
{Ń}{{\'N}}1
}
\begin{document}
\section{Grafy}
\lstinputlisting{grafy.hpp}
\lstinputlisting{grafy.cpp}
\newpage
\section{Tekst}
\lstinputlisting{tekst.hpp}
\lstinputlisting{tekst.cpp}
\newpage
\section{Geometria}
\lstinputlisting{geom.hpp}
\lstinputlisting{geom.cpp}
\newpage
\section{Simplex}
\lstinputlisting{simplex.hpp}
\newpage
\section{Tablica mieszająca}
\lstinputlisting{hash.hpp}
\lstinputlisting{hash.cpp}
\newpage
\section{Automat skończony}
\lstinputlisting{automat.cpp}
\newpage
\section{Kombinatoryka}
\lstinputlisting{kombi.cpp}
\newpage
\section{Minimalne transpozycje}
\lstinputlisting{transpozycje.cpp}
\newpage
\section{Drzewo sufiksowe}
\lstinputlisting{stree.cpp}
\newpage
\section{Teoria gier}
\lstinputlisting{gry.hpp}
\lstinputlisting{gry.cpp}
\section{Teorioliczbowe}
\lstinputlisting{mod.cpp}
\lstinputlisting{bignum.cpp}
\end{document}
